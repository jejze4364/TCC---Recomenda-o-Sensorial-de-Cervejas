\phantomsection
\subsection*{14B. Scottish Heavy}
\addcontentsline{toc}{subsection}{14B. Scottish Heavy}
\textbf{Impressão Geral}: Uma cerveja maltada de baixo teor alcoólico com sabores leves de caramelo, tosta, toffee e frutas. Uma leve secura proveniente de torra compensa a doçura residual no final, com o amargor percebido apenas para evitar que a cerveja seja enjoativa.

\textbf{Aroma}: Malte de médio-baixo a médio com notas de caramelo e toffee, e qualidades levemente tostadas e açucaradas que podem remeter a migalhas de pão torrado, biscoito champagne, biscoitos ingleses, bolacha crackers ou caramelo. Leve frutado de maçã/pera e leve aroma de lúpulo inglês (terroso, floral, laranja-cítrico, picante, etc.) são permitidos.

\textbf{Aparência}: De cobre claro a marrom. Límpida. Colarinho cremoso, quase branco, de baixo a moderado.

\textbf{Sabor}: Malte médio como pão tostado, com nuances de caramelo e toffee, terminando com uma leve secura proveniente de torra. Uma ampla gama de sabores de açúcar caramelizado e pão tostado é possível, usando descritores semelhantes ao aroma. Maltado e perfil de fermentação limpos. Ésteres e sabor de lúpulo leves são permitidos (descritores semelhantes ao aroma). Amargor suficiente para não ser enjoativa, mas com equilíbrio e retrogosto de maltado.

\textbf{Sensação na Boca}: Corpo médio-leve a médio. Carbonatação de baixa a moderada. Pode ser moderadamente cremosa.

\textbf{Comentários}: Veja a introdução da categoria para comentários detalhados. Pode não parecer tão amarga quanto as especificações indicam devido à densidade final maior e dulçor residual. Não confunda  a leve secura proveniente de torra com defumado; o defumado não está presente nessas cervejas.

\textbf{História}: Veja a introdução da categoria. Os nomes cervejas Shilling eram usados para cervejas suaves (não envelhecidas) antes da Primeira Guerra Mundial, mas os estilos assumiram uma forma moderna apenas após a Segunda Guerra Mundial.

\textbf{Ingredientes}: Na sua forma mais simples, malte pale ale e malte com mais cor, mas também pode usar açúcares, milho, trigo, malte crystal, corantes e uma variedade de outros grãos. Levedura limpa. Água leve. Sem malte defumado com turfa.

\textbf{Comparação de Estilos}: Veja a introdução da categoria. Semelhante a outras Scottish Ales em perfil de sabor, de cor mais clara e mais forte que uma Scottish Light. Semelhante em teor alcoólico a Ordinary Bitter, mas com um perfil de sabor e equilíbrio diferentes.

\begin{tabular}{@{}p{35mm}p{35mm}@{}}
  \textbf{Estatísticas}: & OG: 1,035 - 1,040 \\
  IBU: 10 - 20 & FG: 1,010 - 1,015  \\
  SRM: 12 - 20  & ABV: 3,3\% - 3,9\%
\end{tabular}

\textbf{Exemplos Comerciais}: McEwan's 70, Orkney Raven Ale.

\textbf{Última Revisão}: Scottish Heavy (2015)

\textbf{Atributos de Estilo}: amber-ale-family, amber-color, british-isles, malty, session-strength, top-fermented, traditional-style

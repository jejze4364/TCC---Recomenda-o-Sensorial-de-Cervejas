\subsection*{Inscrevendo Cervejas de Especialidade}
\addcontentsline{toc}{subsection}{Inscrevendo Cervejas de Especialidade}
Muitos cervejeiros têm dúvidas sobre onde inscrever suas Cervejas de Especialidade e a melhor forma de descrevê-las. Siga estas sugestões para um melhor resultado:
\subsubsection*{Instruções para Inscrição}
Inscrever uma Cerveja de Especialidade em uma competição requer mais informações do que apenas selecionar o estilo. Examine a sessão de \textit{Instruções para Inscrição} dentro do estilo selecionado para as informações específicas obrigatórias. Juízes necessitam destas informações e não podem julgar apropriadamente a sua cerveja sem elas. Sua nota será prejudicada se elas forem omitidas.

Quando estiver decidindo quais informações opcionais fornecer, imagine-se na posição dos juízes. Forneça informações pertinentes que irão ajudar eles a entender sua cerveja e seu propósito. Evite informações inúteis e irrelevantes que não ajude os juízes a entender sua cerveja. Não utilize descrições exageradas de marketing/vendas. Não utilize nenhuma informação que possa fazer com que os juízes determinem sua identidade. Alguns softwares de competição limitam o tamanho dos comentários, então escolha suas palavras com cuidado.

\subsubsection*{Estilo Base}
A maioria das Cervejas de Especialidade necessita que um Estilo Base seja identificado ou ao menos uma descrição da cerveja – veja as \textit{Instruções para Inscrição} dos estilos para o que é necessário. Se é obrigatório declarar um Estilo Base, use algum dos estilos nomeados nas Categorias de 1 a 27, incluindo cervejas de estilos ou categorias com alternativas enumeradas (como \textit{Historical} ou \textit{Specialty IPA}). Estilos Provisórios do website do BJCP e Estilos Locais do Apêndice B também podem ser usados como Estilo Base.

Se as Instruções para Inscrição permitem que uma família de estilos genérica pode ser usada, isto quer dizer um estilo amplo no senso geral – como IPA, Porter ou Stout. Não é necessário que você diga qual o tipo de Porter, por exemplo, mas você deve dar uma descrição geral da cerveja. Algumas cervejas que são projetadas para demonstrar algum ingrediente especial possuem bases bastante neutras.

Não utilize Cerveja de Especialidade como Estilo Base de outra Cerveja de Especialidade, a menos que as Instruções para Inscrição para aquele estilo permitam explicitamente. Diversas categorias de Estilos de Especialidade possuem um estilo ‘de Especialidade’ que permite certos Ingredientes de Especialidade. Do contrário, o estilo de cerveja 34B ‘Mixed-Style’ pode ser usado.

\subsubsection*{Ingredientes de Especialidade/Especiais}
Quanto mais especifico ou extravagante você for na descrição de seu ingrediente especial, mais os juízes irão procurar por este caráter. Prove sua cerveja e então destaque os ingredientes que são identificáveis. Se apenas um ingrediente de especialidade foi utilizado, ele deve contribuir com uma característica identificável para a cerveja. Caso você mencione múltiplos ingredientes, eles não precisam ser todos individualmente identificáveis, mas devem contribuir para a experiência sensorial como um todo.

Caso você mencione um ingrediente fora do comum, você talvez queira descrever seu caráter, ou ao menos verificar se uma busca na internet pelo seu nome trará uma referência útil para os juízes. Fornecer um termo de busca é uma boa alternativa.

Um nome simples ou genérico do ingrediente é normalmente a melhor escolha, a menos que seu ingrediente tenha um perfil incomum. Caso você use uma combinação de ingredientes como, por exemplo, condimentos, normalmente você pode se referir ao blend pelo seu nome comum (por exemplo, pumpkin pie spice/condimentos de torta de abóbora, pó de curry) ao invés de cada condimento de forma individual.

\textit{(Nota do Tradutor: para um exemplo mais local, especiarias de doce de abóbora - composto normalmente por canela, cravo e gengibre).}

Caso você utilize algum ingrediente que seja potencialmente alergênico, sempre o declare, mesmo que não seja possível sua percepção.

Exemplo: “alergênico: amendoim” – juízes não devem penalizar uma cerveja quando um alergênico declarado não é percebido.

\subsubsection*{Melhor Enquadramento}
Inscrever uma cerveja com um Ingrediente de Especialidade apenas e um Estilo Base Clássico é algo óbvio. Escolher o melhor estilo para a cerveja com uma combinação de Ingredientes de Especialidade requer um pouco de análise. Quando estiver escolhendo o estilo no qual sua Cerveja de Especialidade será inscrito, procure pelo melhor enquadramento dentre todas as possíveis alternativas em que a combinação dos ingredientes é permitida. Escolhe um estilo que represente o ingrediente dominante ou, caso os ingredientes estejam em equilíbrio, selecione o primeiro Estilo de Especialidade na qual ela se enquadre.

Inscrever uma cerveja em um Estilo de Especialidade é uma mensagem aos juízes que sua cerveja possui certos elementos identificáveis. Caso você tenha utilizado um ingrediente, mas ele não pode ser percebido, então não inscreva em um estilo que requer aquele ingrediente. Se os juízes não conseguem identificar algo, eles vão considerar que está ausente e reduzirão a nota de acordo.

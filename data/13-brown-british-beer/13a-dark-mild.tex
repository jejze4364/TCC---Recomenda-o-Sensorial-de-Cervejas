\phantomsection
\subsection*{13A. Dark Mild}
\addcontentsline{toc}{subsection}{13A. Dark Mild}
\textbf{Impressão Geral}: Uma \textit{session} ale britânica escura, de baixa densidade e focada em malte, adequada para beber em quantidade. Refrescante, mas saborosa para seu teor alcoólico, com uma ampla gama de expressão de malte escuro ou açúcar escuro.

\textbf{Aroma}: Aroma de malte de baixo a moderado, podendo ter algum sabor frutado. A expressão do malte pode assumir uma ampla gama de caráter, que pode incluir caramelo, \textit{toffee}, cereais, tostado, nozes, chocolate ou levemente torrado. Aroma baixo de lúpulo terroso ou floral é opcional. Diacetil muito baixo é opcional.

\textbf{Aparência}: Cor de cobre a marrom escuro ou mogno. Geralmente limpa, embora tradicionalmente não seja filtrada. Colarinho de baixo a moderado, com cor de quase branco a castanho; a retenção pode ser ruim.

\textbf{Sabor}: Geralmente uma cerveja maltada, embora possa ter uma ampla variedade de sabores à base de malte e de levedura (por exemplo, maltado, doce, caramelo, \textit{toffee}, tostado, nozes, chocolate, café, torrado, frutas, alcaçuz, ameixa, passas) sobre uma base de pão, biscoito ou tostado. Pode terminar de doce a seca. Versões com maltes mais escuros podem ter um final seco e torrado. Amargor de baixo a moderado, o suficiente para fornecer algum equilíbrio, mas não o suficiente para dominar o malte no equilíbrio. Ésteres frutados moderados são opcionais. Sabor de lúpulo baixo é opcional. Diacetil baixo é opcional.

\textbf{Sensação na Boca}: Corpo de leve a médio. Carbonatação geralmente de baixa a média-baixa. As versões tostadas podem ter uma leve adstringência. Versões mais doces podem parecer ter uma sensação na boca bastante cheia para a densidade. Não deve ser sem carbonatação, aguada ou sem corpo.

\textbf{Comentários}: A maioria das cervejas é \textit{session}, com baixo teor alcoólico, em torno de 3,2\%, embora algumas versões possam ser feitas em uma faixa de teor alcoólico mais alto (4\% ou mais) para exportação, festivais, ocasiões sazonais ou especiais. Geralmente servidas de barris (\textit{casks}); versões \textit{session} e engarrafadas tendem a não suportar bem o transporte. Uma ampla gama de interpretações é possível. Existem versões claras (âmbar médio a marrom claro), mas estas são ainda mais raras do que as escuras; essas diretrizes descrevem apenas a versão escura moderna.

\textbf{História}: Historicamente, \textit{mild} era simplesmente uma cerveja não envelhecida e poderia ser usada como um adjetivo para distinguir entre cervejas envelhecidas ou mais lupuladas. As \textit{milds} modernas têm suas raízes nas cervejas tipo X mais fracas dos anos 1800, que começaram a ficar mais escuras na década de 1880, mas somente após a Primeira Guerra Mundial elas se tornaram marrom escuras. No uso atual, o termo implica uma cerveja de baixa intensidade com menos amargor de lúpulo do que as \textit{bitters}. As diretrizes descrevem a versão britânica moderna. O termo \textit{mild} está atualmente um pouco em desuso entre os consumidores e muitas cervejarias não o usam mais. Cada vez mais raro. Não há conexão histórica ou relação entre Mild e Porter.

\textbf{Ingredientes}: Maltes base Pale britânicos (geralmente com bastante dextrina), maltes Crystal, maltes escuros ou açúcares escuros como adjuntos, também podem incluir adjuntos como milho em flocos e podem ser escurecidas com caramelo cervejeiro. Levedura ale britânica característica. Qualquer tipo de lúpulo, já que seu caráter é escondido e raramente é perceptível.

\textbf{Comparação de Estilos}: Algumas versões podem parecer Porters inglesas modernas de baixa densidade. Muito menos doce que a London Brown Ale.

\begin{tabular}{@{}p{35mm}p{35mm}@{}}
  \textbf{Estatísticas}: & OG: 1,030 - 1,038 \\
  IBU: 10 - 25  & FG: 1,008 - 1,013  \\
  SRM: 14 - 25  & ABV: 3\% - 3,8\%
\end{tabular}

\textbf{Exemplos Comerciais}: Brain's Dark, Greene King XX Mild, Hobson's Champion Mild, Mighty Oak Oscar Wilde, Moorhouse Black Cat, Theakston Traditional Mild.

\textbf{Última Revisão}: Dark Mild (2015)

\textbf{Atributos de Estilo}: british-isles, brown-ale-family, dark-color, malty, session-strength, top-fermented, traditional-style

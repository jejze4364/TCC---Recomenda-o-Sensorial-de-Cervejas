\phantomsection
\subsection*{27A. Historical Beer: Sahti}
\addcontentsline{toc}{subsection}{27A. Historical Beer: Sahti}
\textbf{Impressão Geral}: Uma cerveja finlandesa \textit{farmhouse} tradicional forte, pesada e doce, geralmente com centeio e zimbro, e um caráter de levedura de banana e cravo.

\textbf{Aroma}: Impressão de malte doce, como mosto. Maltado como cereais, caramelo e centeio em segundo plano. Leve aroma de álcool. Éster alto como banana com fenóis de moderados a moderadamente altos como cravo. Pode ter um caráter amadeirado de zimbro de leve a moderado. Não é azeda. Sem lúpulo.

\textbf{Aparência}: Cor de amarelo a marrom escuro; a maioria é de âmbar médio a âmbar escuro. Geralmente um pouco nebulosa e turva. Colarinho pequeno, devido à baixa carbonatação.

\textbf{Sabor}: Bastante doce e muitas vezes com sabor de malte cru como mosto, como cereais com algum caramelo e \textit{toffee}. Amargor baixo. Sem sabor de lúpulo. Caráter baixo amadeirado ou como pinho é aceitável. Éster de moderado a alto como banana e frutado, cravo e condimentado moderados. Final bastante doce. Fresca, não é azeda.

\textbf{Sensação na Boca}: Espessa, viscosa e pesada com proteínas (sem fervura significa sem \textit{hot break}). De quase sem carbonatação a moderadamente alta, similar a Cask Ales inglesas. Aquecimento pelo teor alcoólico e pelo frescor, porém muitas vezes mascarado pelo dulçor.

\textbf{Comentários}: O uso do centeio não significa que deve ter sabor como cominho (um sabor comum em pães de centeio). O zimbro atua um pouco como o lúpulo no equilíbrio e sabor, provendo um contraponto de sabor e amargor ao dulçor de malte. Caráter de zimbro amadeirado e como pinho é mais comum do que o frutado como gin.

\textbf{História}: Um estilo tradicional nativo da Finlândia; uma tradição \textit{farmhouse} de pelo menos 500 anos, geralmente feito durante ocasiões festivas como casamentos de verão e consumidas dentro de uma ou duas semanas após a brassagem.

\textbf{Ingredientes}: Cevada malteada. Centeio é comum. Lúpulo baixo, se tiver. Galhos de zimbro (com ou sem bagas) são utilizados para filtração (tradicionalmente troncos ocos de madeira). Utilização de fermento finlandês para produção de pão numa fermentação quente e rápida (levedura alemã de Weizen é uma substituição razoável). Longo regime de mosturação. O mosto não é fervido.

\textbf{Comparação de Estilos}: Traz semelhanças com Weizenbocks, porém doce e espessa, com caráter de centeio e zimbro.

\begin{tabular}{@{}p{35mm}p{35mm}@{}}
  \textbf{Estatísticas}: & OG: 1,076 - 1,120 \\
  IBU: 0 - 15  & FG: 1,016 - 1,038  \\
  SRM: 4 - 22 & ABV: 7\% - 11\%
\end{tabular}

\textbf{Exemplos Comerciais}: Agora é feita na Finlândia por diversas cervejarias durante o ano.

\textbf{Última Revisão}: Historical Beer: Sahti (2015)

\textbf{Atributos de Estilo}: amber-color, central-europe, high-strength, historical-style, spice, top-fermented
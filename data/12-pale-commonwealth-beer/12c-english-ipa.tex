\phantomsection
\subsection*{12C. English IPA}
\addcontentsline{toc}{subsection}{12C. English IPA}
\textbf{Impressão Geral}: Uma ale britânica clara, amarga, com teor alcoólico moderado, muito bem atenuada, com final seco e aroma e sabor lupulados. Os ingredientes britânicos clássicos fornecem o perfil de sabor mais autêntico.

\textbf{Aroma}: Um aroma de lúpulo de moderado a moderadamente alto, tipicamente floral, picante condimentado ou cítrico, de laranja natural. Um leve aroma de \textit{dry-hopping} é aceitável, mas não obrigatório. Malte como pão ou biscoito de médio-baixo a médio, opcionalmente com presença de malte como caramelo ou tostado moderadamente baixo. Frutado de baixo a moderado é aceitável. Nota sulfurosa leve opcional.

\textbf{Aparência}: A cor varia de dourado a âmbar profundo, mas a maioria é bastante clara. Deve ser límpida, embora as versões não filtradas com \textit{dry-hopping} possam ser um pouco turvas. Colarinho persistente, de tamanho moderado e com cor quase branca.

\textbf{Sabor}: Sabor de lúpulo de médio a alto, com amargor de lúpulo de moderado a assertivo. O sabor do lúpulo deve ser semelhante ao aroma (floral, picante condimentado ou cítrico de laranja). O sabor do malte deve ser de médio-baixo a médio, algo como pão, opcionalmente com aspectos de biscoito de leve a médio-leve, torrada, \textit{toffee} ou caramelo. Frutado de médio-baixo a médio. O final é de meio seco a muito seco e o amargor pode permanecer no retrogosto, mas não deve ser áspero. O equilíbrio é em direção ao lúpulo, mas o malte ainda deve ser perceptível no suporte. Se for usada água com alto teor de sulfato, proporcionará um final distintamente mineral e seco, algum sabor sulfuroso e um amargor persistente. Algum sabor de álcool limpo pode ser notado em versões com maior teor alcoólico.

\textbf{Sensação na Boca}: Suave, corpo de médio-leve a médio, sem adstringência derivada do lúpulo. A carbonatação de média a média-alta pode dar uma sensação geral seca, apesar do suporte do malte. Um aquecimento alcoólico baixo e suave do álcool pode ser sentido em versões com maior teor alcoólico.

\textbf{Comentários}: Os atributos da IPA que foram importantes para sua chegada em boas condições na Índia foram que ela era muito bem atenuada e fortemente lupulada. Simplesmente porque foi assim que a IPA foi enviada, não significa que outras cervejas como a Porter também não foram enviadas para a Índia, que a IPA foi inventada para ser enviada para a Índia, que a IPA foi mais lupulada do que outras cervejas de guarda ou que o nível de álcool era incomum para a época. Muitos exemplos modernos rotulados como IPA tem baixo teor alcoólico. De acordo com a CAMRA, “as chamadas IPAs teor alcoólico de cerca de 3,5\% não são fiéis ao estilo”. O historiador de cerveja inglês Martyn Cornell comentou que cervejas como essa “não são realmente distinguíveis de ordinary bitters”. Portanto, optamos por concordar com essas fontes para nossas diretrizes, em vez do que algumas cervejarias britânicas modernas estão chamando de IPA; basta estar ciente desses dois principais tipos de IPAs no mercado britânico hoje. As cervejas eram embarcadas em barris de carvalho bem usados, então o estilo não deveria ter um caráter de carvalho (madeira) ou Brett.

\textbf{História}: Originalmente uma \textit{stock ale} clara de Londres que foi enviada pela primeira vez para a Índia no final dos anos 1700. George Hodgson da Bow Brewery não criou o estilo, mas foi o primeiro cervejeiro conhecido a dominar o mercado. Após uma disputa comercial, a Companhia das Índias Orientais fez Samuel Allsopp recriar (e reformular) a cerveja em 1823 usando a água rica em sulfato de Burton. O nome India Pale Ale não foi usado até por volta de 1830. O teor alcoólico e a popularidade diminuíram com o tempo, e o estilo praticamente desapareceu na segunda metade do século XX. Enquanto a IPA com maior teor alcoólico do tipo Burton permaneceu, o nome também foi aplicado a produtos lupulados, de baixa densidade, muitas vezes engarrafados (uma tendência que continua em alguns exemplos britânicos modernos). O estilo passou por uma redescoberta da cerveja artesanal na década de 1980, e é o que está descrito nestas diretrizes. Exemplos modernos são inspirados em versões clássicas, mas não se deve presumir que tenham uma linhagem ininterrupta com exatamente o mesmo perfil. White Shield é provavelmente o exemplo com a linhagem mais longa, remontando às Burton IPAs antigas com maior teor alcoólico, fabricadas pela primeira vez em 1829.

\textbf{Ingredientes}: Malte pale ale. Lúpulos ingleses, particularmente como lúpulos de finalização. Levedura ale britânica de alta atenuação. Açúcar refinado pode ser usado em algumas versões. Água com caráter de sulfato opcional do tipo Burton.

\textbf{Comparação de Estilos}: Geralmente terá mais lúpulos tardios e menos frutado e caramelo do que as British Pale Ales e Bitters. Tem menos intensidade de lúpulo e um sabor de malte mais pronunciado do que as típicas American IPAs.

\begin{tabular}{@{}p{35mm}p{35mm}@{}}
  \textbf{Estatísticas}: & OG: 1,050 - 1,070 \\
  IBU: 40 - 60  & FG: 1,010 - 1,015  \\
  SRM: 6 - 14  & ABV: 5\% - 7,5\%
\end{tabular}

\textbf{Exemplos Comerciais}: Berkshire Lost Sailor IPA, Fuller's Bengal Lancer IPA, Marston’s Old Empire IPA, Meantime London IPA, Thornbridge Jaipur, Worthington White Shield.

\textbf{Última Revisão}: English IPA (2015)

\textbf{Atributos de Estilo}: bitter, british-isles, high-strength, hoppy, ipa-family, pale-color, top-fermented, traditional-style

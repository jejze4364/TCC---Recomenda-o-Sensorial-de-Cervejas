\phantomsection
\subsection*{4B. Festbier}
\addcontentsline{toc}{subsection}{4B. Festbier}
\textbf{Impressão Geral}: Uma cerveja lager alemã clara, suave, limpa, com sabor maltado moderadamente forte e um leve caráter de lúpulo. Equilibra com perfeição força e facilidade de consumo, com uma impressão na boca e final que incentiva o consumo. Destaca os elegantes sabores de malte alemão sem se tornar muito pesada ou encorpada.

\textbf{Aroma}: Maltado rico moderado com uma ênfase em aromas tostados e/ou de massa de pão e uma impressão de dulçor. Lúpulo floral, herbal e/ou condimentado de baixo a médio-baixo. O malte não deve ter uma qualidade altamente tostada, de caramelo ou tipo biscoito. Perfil de fermentação de lager limpo.

\textbf{Aparência}: Cor de amarelo profundo a dourado profundo, não deve ter tons âmbar. Limpidez brilhante. Espuma persistente com cor de branca a quase branca. A maior parte dos exemplares comerciais são de coloração dourado claro.

\textbf{Sabor}: Sabor inicial maltado de médio a médio-alto, com um caráter de massa de pão levemente tostada e uma impressão suave de um maltado rico. Amargor de médio a médio-baixo, definitivamente maltada no equilíbrio. Bem atenuada e bem definida, mas não seca. Sabor de lúpulo floral, herbal e/ou condimentado de médio-baixo a médio. Perfil de fermentação limpo. O sabor é principalmente de malte Pils, mas com leves notas tostadas. O amargor é para dar suporte, mas ainda assim deve apresentar um final maltado e saboroso.

\textbf{Sensação na Boca}: Corpo médio com uma textura suave, levemente cremosa. Carbonatação média. O álcool é quase imperceptível e, se notado, surge apenas como um sutil aquecimento.

\textbf{Comentários}: Esse estilo representa a cerveja alemã moderna servida na Oktoberfest (embora não somente para a Oktoberfest, pode ser encontrada em várias outras festas) e às vezes é chamada de Wiesn (“a campina”, o nome do local da Oktoberfest). Nós escolhemos chamar esse estilo de \textit{Festbier} dado as leis alemãs e europeias já que \textit{Oktoberfestbier} tem denominação de origem e só pode ser usado para as cervejas produzidas nas cervejarias de grande porte dentro dos limites da cidade de Munique e consumidas na Oktoberfest. Outros países não estão vinculados a essas regras, então muitas cervejarias artesanais nos Estados Unidos produzem cervejas chamadas Oktoberfest, mas baseadas no estilo tradicional descrito nesse guia como Märzen. Pode ser chamada de Helles Märzen.

\textbf{História}: Desde 1990, a maior parte das cervejas servidas na Oktoberfest de Munique são desse estilo. As cervejas exportadas especificamente para os Estados Unidos ainda são do estilo âmbar tradicional, assim como as interpretações produzidas nos Estados Unidos. A Paulaner criou a primeira versão dourada em meados de 1970 porque eles acharam que a Oktoberfest tradicional era muito encorpada. Então eles desenvolveram uma versão mais leve, mais fácil de beber, mas ainda maltada, com o objetivo de ser “mais fácil de consumir em grandes quantidades” (de acordo com o cervejeiro chefe da Paulaner). Mas o real tipo de cerveja servido na Oktoberfest é definido por um comitê da cidade de Munique.

\textbf{Ingredientes}: Principalmente malte Pils, mas com um pouco de malte Vienna ou Munich para aumentar o maltado. As diferenças nos exemplares comerciais são principalmente devido ao uso de maltes de diferentes maltarias e diferentes leveduras, e não a diferenças significativas na composição de cereais.

\textbf{Comparação de Estilos}: Menos intensa e menos ricamente tostada do que uma Märzen. Mais forte que uma Munich Helles, com um pouco mais de corpo e sabor de lúpulo e malte. Menos rica em intensidade de malte do que uma Helles Bock. A complexidade de malte é similar à de uma Czech Premium Pale Lager com uma densidade mais alta, embora sem os lúpulos associados.

\begin{tabular}{@{}p{35mm}p{35mm}@{}}
  \textbf{Estatísticas}: & OG: 1,054 - 1,057 \\
  IBU: 18 - 25  & FG: 1,010 - 1,012  \\
  SRM: 4 - 6   & ABV: 5,8\% - 6,3\%
\end{tabular}

\textbf{Exemplos Comerciais}: Augustiner Oktoberfest, Hacker-Pschorr Superior Festbier, Löwenbräu Oktoberfestbier, Hofbräu Oktoberfestbier, Paulaner Oktoberfest Bier, Weihenstephaner Festbier.

\textbf{Última Revisão}: Festbier (2015)

\textbf{Atributos de Estilo}: bottom-fermented, central-europe, lagered, malty, pale-color, pale-lager-family, standard-strength, traditional-style.

\phantomsection
\subsection*{26C. Belgian Tripel}
\addcontentsline{toc}{subsection}{26C. Belgian Tripel}
\textbf{Impressão Geral}: Uma cerveja belga forte, clara, um tanto condimentada, com um agradável sabor arredondado de malte, amargor assertivo e final seco. Bastante aromática, com condimentado, frutado e notas de álcool combinando com o caráter maltado limpo pra produzir uma cerveja surpreendentemente agradável de ser bebida, apesar do alto teor alcoólico.

\textbf{Aroma}: Buquê complexo, porém harmonioso, de condimentado com intensidade de moderada a significativa, ésteres frutados moderados, percepção alcoólica baixa, lúpulagem baixa e maltado baixo. Generosos fenóis condimentados, apimentados e às vezes semelhantes a cravo. Ésteres frequentemente remetem a frutas cítricas, como laranjas ou limões, mas às vezes podem ter leve caráter de banana madura. Um caráter de lúpulo baixo, porém distinto, condimentado, floral e às vezes perfumado é opcional. Álcoois são suaves, condimentados e de intensidade baixa. O caráter de malte é leve, com uma impressão suave, levemente doce como grãos ou com leve toque de mel.

\textbf{Aparência}: A cor varia de amarelo profundo a âmbar claro. Boa limpidez. Efervescente. Espuma branca de longa duração, cremosa, resultando no característico “rendado belga” nas bordas do copo.

\textbf{Sabor}: Perfil similar ao do aroma (se aplicam os mesmos descritores) para malte, ésteres, fenóis, álcool e lúpulos. Ésteres são de baixos a moderados, fenóis de baixos a moderados, lúpulos de baixos a moderados; álcool baixo, todos bem combinados em uma apresentação coerente. O amargor é de médio a alto, acentuado por um final seco. No retrogosto, o amargor é moderado com um caráter substancial condimentado e frutado oriundo da levedura. Não deve ser doce.

\textbf{Sensação na Boca}: O corpo é de médio-baixo a médio, porém mais leve do que a densidade possa sugerir. Altamente carbonatada. O teor alcoólico é enganoso e pode ter uma pequena, mas não clara sensação de aquecimento. Efervescente. Não deve ser pesada.

\textbf{Comentários}: O teor alcoólico é alto, mas não tem gosto forte de álcool. Os melhores exemplos são enganosos, não óbvios. A carbonatação e a atenuação altas ajudam a realçar os muitos sabores e reforçam a percepção de um final seco. As versões tradicionais possuem ao menos 30 de IBU e são muito secas.

\textbf{História}: Popularizado pelo monastério de Westmalle, que o produziu em 1931 pela primeira vez.

\textbf{Ingredientes}: Malte Pilsen, sendo muitas vezes usados açúcares claros como adjuntos. Lúpulos continentais. Cepas de leveduras frutadas e condimentadas. Adições de especiarias geralmente não são tradicionais e, se utilizadas, devem ser discretas. Água relativamente mole.

\textbf{Comparação de Estilos}: Pode ser semelhante a uma Belgian Golden Strong Ale, mas um pouco mais escura e um pouco mais encorpada, com ênfase nos fenóis e menos em ésteres, e menos lúpulos com adição tardia. Não deve parecer com uma Barleywine versão clara.

\begin{tabular}{@{}p{35mm}p{35mm}@{}}
  \textbf{Estatísticas}: & OG: 1,075 - 1,085 \\
  IBU: 20 - 40  & FG: 1,008 - 1,014  \\
  SRM: 4,5 - 7  & ABV: 7,5\% - 9,5\%
\end{tabular}

\textbf{Examples Comerciais}: Chimay Tripel, La Rulles Tripel, La Trappe Tripel, St. Bernardus Tripel, Val-Dieu Triple, Westmalle Tripel.

\textbf{Última Revisão}: Belgian Tripel (2015)

\textbf{Atributos de Estilo}: bitter, high-strength, pale-color, top-fermented, traditional-style, western-europe

\documentclass[12pt,a4paper]{article}

% Pacotes necessários
\usepackage[portuguese]{babel}
\usepackage[margin=3cm,top=3cm,bottom=2cm]{geometry}
\usepackage{amsmath,amssymb,amsfonts}
\usepackage{graphicx}
\usepackage{float}
\usepackage{hyperref}
\usepackage{fancyhdr}
\usepackage{setspace}
\usepackage{titlesec}
\usepackage{tocloft}
\usepackage{indentfirst}
\usepackage{algorithm}
\usepackage{algorithmic}
\usepackage{listings}
\usepackage{xcolor}
\usepackage{booktabs}
\usepackage{array}
\usepackage{multirow}

% Configurações gerais
\onehalfspacing
\setlength{\parindent}{1.25cm}

% Configuração do hyperref
\hypersetup{
    colorlinks=true,
    linkcolor=black,
    citecolor=black,
    urlcolor=blue
}

% Configuração de cabeçalho e rodapé
\pagestyle{fancy}
\fancyhf{}
\fancyhead[R]{\thepage}
\renewcommand{\headrulewidth}{0pt}

% Configuração das seções
\titleformat{\section}{\normalfont\bfseries\large}{\thesection}{1em}{}
\titleformat{\subsection}{\normalfont\bfseries\normalsize}{\thesubsection}{1em}{}
\titleformat{\subsubsection}{\normalfont\bfseries\normalsize}{\thesubsubsection}{1em}{}

% Configuração do sumário
\renewcommand{\cftsecleader}{\cftdotfill{\cftdotsep}}

% Configuração do código Julia
\lstset{
    language=Python,
    basicstyle=\ttfamily\footnotesize,
    commentstyle=\color{gray},
    keywordstyle=\color{blue},
    stringstyle=\color{red},
    showstringspaces=false,
    breaklines=true,
    frame=single,
    numbers=left,
    numberstyle=\tiny\color{gray}
}

\begin{document}

% ========== CAPA ==========
\begin{titlepage}
\begin{center}
\large
\textbf{UNIVERSIDADE FEDERAL FLUMINENSE}\\
\textbf{ESCOLA DE ENGENHARIA}\\
\textbf{CURSO DE ENGENHARIA DE PRODUÇÃO}

\vspace{4cm}

\Large
\textbf{JOSÉ EDUARDO QUICOLI FONSECA}

\vspace{3cm}

\Large
\textbf{DESENVOLVIMENTO DE UM ALGORITMO DE RECOMENDAÇÃO DE CERVEJAS PERSONALIZADAS UTILIZANDO A LINGUAGEM JULIA}

\vspace{3cm}

\large
Trabalho de Conclusão de Curso apresentado ao Curso de\\
Engenharia de Produção da Universidade Federal Fluminense\\
como requisito parcial para obtenção do grau de\\
Engenheiro de Produção.

\vspace{2cm}

\begin{flushleft}
\large
Orientador: Prof. Dr. Artur Alves Pessoa\\
Co-orientador: \underline{\hspace{6cm}}
\end{flushleft}

\vfill

\large
\textbf{NITERÓI}\\
\textbf{2025}
\end{center}
\end{titlepage}

% ========== FOLHA DE APROVAÇÃO ==========
\newpage
\begin{center}
\Large
\textbf{FOLHA DE APROVAÇÃO}
\end{center}

\vspace{2cm}

Trabalho de Conclusão de Curso apresentado por \textbf{José Eduardo Quicoli Fonseca}, intitulado \textbf{"Desenvolvimento de um Algoritmo de Recomendação de Cervejas Personalizadas Utilizando a Linguagem Julia"}, aprovado em \underline{\hspace{1cm}}/\underline{\hspace{1cm}}/\underline{\hspace{2cm}}, pela banca examinadora composta por:

\vspace{4cm}

\begin{center}
\underline{\hspace{8cm}}\\
Prof. Dr. Artur Alves Pessoa\\
Orientador - Universidade Federal Fluminense

\vspace{2cm}

\underline{\hspace{8cm}}\\
Prof. Dr. \underline{\hspace{4cm}}\\
Membro da Banca

\vspace{2cm}

\underline{\hspace{8cm}}\\
Prof. Dr. \underline{\hspace{4cm}}\\
Membro da Banca
\end{center}

% ========== RESUMO ==========
\newpage
\begin{center}
\Large
\textbf{RESUMO}
\end{center}

\vspace{1cm}

O presente Trabalho de Conclusão de Curso tem como objetivo o desenvolvimento de um algoritmo de recomendação de cervejas personalizadas utilizando a linguagem de programação Julia. A proposta se fundamenta na crescente demanda por sistemas inteligentes que proporcionem experiências de consumo mais adequadas ao perfil individual dos usuários, especialmente no mercado de cervejas artesanais. Para isso, foi construída uma base de dados contendo informações técnicas e sensoriais de diferentes estilos de cerveja, como teor alcoólico (ABV), amargor (IBU), cor (SRM) e perfil de sabor. As preferências dos usuários foram coletadas por meio de um questionário estruturado, e utilizadas como entrada em um sistema de filtragem baseado em conteúdo. O algoritmo desenvolvido permite o ajuste dinâmico dos perfis de preferência a partir do feedback explícito dos usuários, estabelecendo um mecanismo de retroalimentação que melhora progressivamente a acurácia das recomendações. Os testes realizados com perfis simulados demonstraram que, após três ciclos de feedback, houve aumento significativo na aderência das recomendações às preferências individuais. A linguagem Julia se mostrou eficiente para o processamento de dados e execução das operações matemáticas necessárias, especialmente devido ao seu desempenho computacional elevado. Como contribuições futuras, destacam-se a ampliação da base de dados, a implementação de modelos híbridos de recomendação e o desenvolvimento de uma interface gráfica voltada ao usuário final.

\vspace{1cm}

\textbf{Palavras-chave:} Algoritmos de Recomendação. Sistemas de Sugestão. Cervejas Artesanais. Linguagem Julia. Retroalimentação.

% ========== ABSTRACT ==========
\newpage
\begin{center}
\Large
\textbf{ABSTRACT}
\end{center}

\vspace{1cm}

This Final Undergraduate Project aims to develop a personalized beer recommendation algorithm using the Julia programming language. The proposal is based on the growing demand for intelligent systems that provide consumption experiences tailored to individual preferences, particularly in the craft beer market. A database was built containing technical and sensory attributes of different beer styles, such as alcohol by volume (ABV), bitterness (IBU), color (SRM), and flavor profile. User preferences were collected through a structured questionnaire and served as input for a content-based filtering system. The developed algorithm dynamically adjusts user profiles based on explicit feedback, implementing a feedback loop that progressively enhances the accuracy of recommendations. Tests conducted with simulated user profiles demonstrated a significant improvement in recommendation alignment after three feedback cycles. Julia proved to be efficient for data processing and mathematical operations, especially due to its high computational performance. Future contributions include expanding the beer database, implementing hybrid recommendation models, and developing a user-friendly graphical interface.

\vspace{1cm}

\textbf{Keywords:} Recommendation Algorithms. Suggestion Systems. Craft Beers. Julia Language. Feedback.

% ========== LISTA DE FIGURAS ==========
\newpage
\listoffigures

% ========== LISTA DE TABELAS ==========
\newpage
\listoftables

% ========== SUMÁRIO ==========
\newpage
\tableofcontents

% ========== INTRODUÇÃO ==========
\newpage
\section{INTRODUÇÃO}

O mercado de cervejas artesanais tem apresentado crescimento exponencial nas últimas décadas, com consumidores cada vez mais exigentes e interessados em experiências sensoriais diferenciadas. Segundo dados do Ministério da Agricultura, Pecuária e Abastecimento (MAPA, 2020), o Brasil possui mais de 1.200 cervejarias registradas, evidenciando a diversidade e complexidade do mercado nacional.

Neste contexto, a personalização emerge como um diferencial competitivo fundamental. A capacidade de recomendar produtos alinhados às preferências individuais dos consumidores não apenas melhora a experiência de consumo, mas também fortalece a relação entre marca e cliente. Sistemas de recomendação inteligentes têm se mostrado eficazes em diversos domínios, desde e-commerce até entretenimento, demonstrando seu potencial para aplicação no setor cervejeiro.

A linguagem de programação Julia, desenvolvida para computação científica de alto desempenho, apresenta características ideais para implementação de algoritmos complexos de recomendação. Sua sintaxe intuitiva, combinada com velocidade de execução comparável a linguagens compiladas, torna-a uma escolha estratégica para processamento eficiente de dados sensoriais e cálculos matemáticos intensivos.

\subsection{Justificativa}

O desenvolvimento de um sistema de recomendação de cervejas personalizadas justifica-se por diversos fatores. Primeiramente, a crescente complexidade do mercado cervejeiro, com centenas de estilos e milhares de variações, torna a escolha uma tarefa desafiadora para consumidores. Segundo, a natureza subjetiva da experiência sensorial requer abordagens sofisticadas que considerem preferências individuais e sua evolução ao longo do tempo.

Além disso, a implementação em Julia contribui para o avanço do conhecimento científico na área de sistemas de recomendação, demonstrando a aplicabilidade de linguagens modernas de computação científica em problemas reais de negócio. A capacidade de retroalimentação do sistema proposto permite aprendizado contínuo e adaptação às mudanças de preferências dos usuários.

\subsection{Objetivos}

\subsubsection{Objetivo Geral}

Desenvolver um algoritmo de recomendação de cervejas personalizadas com mecanismo de retroalimentação utilizando a linguagem Julia, capaz de adaptar-se dinamicamente às preferências dos usuários.

\subsubsection{Objetivos Específicos}

\begin{itemize}
\item Construir uma base de dados estruturada com características técnicas e sensoriais de diferentes estilos de cerveja;
\item Desenvolver um sistema de coleta de preferências dos usuários através de questionário estruturado;
\item Implementar algoritmos de filtragem baseada em conteúdo para geração de recomendações;
\item Desenvolver mecanismo de retroalimentação para ajuste dinâmico dos perfis de preferência;
\item Validar o sistema através de testes com perfis simulados de usuários;
\item Avaliar o desempenho da linguagem Julia na implementação do sistema proposto.
\end{itemize}

% ========== FUNDAMENTAÇÃO TEÓRICA ==========
\newpage
\section{FUNDAMENTAÇÃO TEÓRICA}

\subsection{A Linguagem Julia}

A linguagem Julia foi desenvolvida por Bezanson et al. (2017) com o objetivo de combinar a facilidade de uso de linguagens interpretadas com o desempenho de linguagens compiladas. Projetada especificamente para computação científica, Julia oferece características únicas que a tornam ideal para implementação de algoritmos de recomendação complexos.

Entre as principais vantagens da linguagem destacam-se: tipagem dinâmica com compilação just-in-time (JIT), suporte nativo a operações vetoriais e matriciais, sistema de múltiplo dispatch que permite especializações eficientes de funções, e sintaxe matematicamente intuitiva. Essas características são particularmente relevantes para sistemas de recomendação que requerem processamento intensivo de dados numéricos e cálculos de similaridade.

O ecossistema Julia inclui pacotes especializados como \texttt{DataFrames.jl} para manipulação de dados estruturados, \texttt{LinearAlgebra.jl} para operações matriciais, e \texttt{MLJ.jl} para aprendizado de máquina, fornecendo as ferramentas necessárias para implementação completa do sistema proposto.

\subsection{Características Sensoriais de Cervejas}

As diretrizes da Brewers Association (2024) estabelecem parâmetros internacionais para categorização de estilos de cerveja com base em atributos técnicos e sensoriais. Os principais atributos utilizados incluem:

\begin{itemize}
\item \textbf{ABV (Alcohol by Volume):} Teor alcoólico da cerveja, expresso em percentual volumétrico;
\item \textbf{IBU (International Bitterness Units):} Medida padronizada do amargor, relacionada à concentração de alfa-ácidos do lúpulo;
\item \textbf{SRM (Standard Reference Method):} Escala de cor da cerveja, variando de 1 (muito clara) a 40+ (muito escura);
\item \textbf{Compostos voláteis:} Substâncias aromáticas como acetaldeído, ésteres (ex: isoamyl acetate), álcoois superiores, diacetil e DMS, que contribuem diretamente para o aroma e sabor final da cerveja;
\item \textbf{Perfil sensorial:} Conjunto de características organolépticas como doçura, corpo, frutado, maltado, tostado, cítrico e floral, frequentemente avaliadas por painéis treinados.
\end{itemize}

Habschied et al. (2023) demonstraram que a análise sensorial, quando combinada com a quantificação de compostos voláteis via cromatografia gasosa acoplada à espectrometria de massas (GC-MS), permite distinguir estilos de cerveja com maior precisão do que métodos físico-químicos isolados. Essa abordagem, conhecida como sensômica, torna possível medir distâncias sensoriais entre amostras a partir de vetores compostos por atributos físico-químicos e voláteis.

Bonatto (2024) reforçou que esses atributos, quando adequadamente normalizados, permitem agrupamento eficaz de estilos similares utilizando mapas auto-organizáveis (SOM). Pereira (2024) validou a aplicabilidade desses parâmetros em algoritmos de aprendizado de máquina, obtendo alta precisão na classificação de estilos.

A incorporação dos dados sensoriais e voláteis proporciona uma base robusta para modelos de recomendação mais precisos, incluindo métodos baseados em distância, como a formulação de p-mediana, onde é possível definir estilos de referência para recomendação com base em proximidade sensorial e perfil desejado do consumidor.

\subsection{Sistemas de Recomendação}

Sistemas de recomendação são ferramentas computacionais que sugerem itens relevantes aos usuários com base em suas preferências e comportamentos passados (Ricci et al., 2015). Esses sistemas tornaram-se fundamentais em diversas aplicações, desde comércio eletrônico até entretenimento digital.

As principais abordagens de recomendação incluem:

\subsubsection{Filtragem Baseada em Conteúdo}

Esta abordagem utiliza atributos dos itens para identificar similaridades com as preferências do usuário. No contexto cervejeiro, características como ABV, IBU e perfil sensorial são comparadas com o perfil de preferências declarado pelo usuário. Pirkola (1998) demonstrou a eficácia dessa abordagem em sistemas de recuperação de informação.

\subsubsection{Filtragem Colaborativa}

Baseia-se na premissa de que usuários com preferências similares no passado terão gostos semelhantes no futuro. Aggarwal (2016) apresenta técnicas avançadas de filtragem colaborativa, incluindo métodos baseados em vizinhança e fatoração de matrizes.

\subsubsection{Modelos Híbridos}

Combinam múltiplas técnicas de recomendação para superar limitações individuais de cada abordagem. Esses modelos têm demonstrado superior desempenho em diversos domínios de aplicação.

Para este trabalho, optou-se pela filtragem baseada em conteúdo devido à sua simplicidade de implementação e eficácia em contextos com dados limitados de histórico de usuários. Pereira (2024) validou essa escolha ao demonstrar que distâncias euclidianas entre características técnicas de cervejas produzem recomendações altamente correlacionadas com preferências declaradas.

\subsection{Algoritmos Certificadores e Retroalimentação}

Carreira (2018) define algoritmos certificadores como sistemas que, além de produzir uma saída computacional, fornecem evidências da validade de seus resultados. Embora este trabalho não implemente certificação formal, emprega princípios similares através de mecanismos de retroalimentação.

A retroalimentação permite ajuste dinâmico dos perfis de preferência baseado nas avaliações dos usuários. Esse processo iterativo melhora progressivamente a acurácia das recomendações, estabelecendo um ciclo virtuoso de aprendizado contínuo.

Bonatto (2024) e Pereira (2024) validaram empiricamente a eficácia de ciclos de retroalimentação em sistemas de recomendação cervejeira, demonstrando aumentos significativos na precisão após três iterações. Pereira (2021) mostrou que mesmo redes neurais simples podem ajustar seus pesos eficientemente baseado em novos exemplos de treinamento.

% ========== METODOLOGIA ==========
\newpage
\section{METODOLOGIA}

\subsection{Revis\~ao Bibliogr\'afica}

A revis\~ao bibliogr\'afica foi conduzida por meio de busca sistem\'atica em bases de dados cient\'ificas como IEEE Xplore, ACM Digital Library, ScienceDirect e Google Scholar. Utilizaram-se os termos "recommendation systems", "beer similarity", "Julia programming", "content-based clustering" e "sensory analysis of beer".

Foram selecionados artigos publicados entre 2015 e 2024, priorizando trabalhos com abordagens quantitativas e metodol\'ogicas relevantes para agrupamento de cervejas com base em atributos sensoriais e f\'isico-qu\'imicos. Tamb\'em foram inclu\'idas as diretrizes oficiais da Brewers Association (2024) e documenta\c{c}\~ao t\'ecnica da linguagem Julia.

\subsection{Coleta de Dados}

A base de dados foi constru\'ida a partir de 50 estilos de cerveja catalogados segundo as diretrizes da Brewers Association (2024). Para cada estilo, foram coletados os seguintes atributos:

\begin{itemize}
    \item \textbf{Caracter\'isticas t\'ecnicas:} ABV, IBU, SRM;
    \item \textbf{Perfil sensorial:} do\c{c}ura, amargor, corpo, frutado, maltado, tostado, c\'itrico, floral (escala 1--10);
    \item \textbf{Compostos vol\'ateis:} presen\c{c}a relativa de \'esteres, \'{a}lcoois superiores, diacetil e outros (dados extra\'idos de GC-MS);
    \item \textbf{Metadados:} nome do estilo, pa\'is de origem, tipo de fermenta\c{c}\~ao.
\end{itemize}

\subsection{Formula\c{c}\~ao p-Mediana}

A metodologia proposta emprega o modelo de p-mediana para defini\c{c}\~ao de estilos de cerveja que melhor representam grupos de prefer\^encias sensoriais. O problema da p-mediana \'{e} formulado como um problema de programa\c{c}\~ao inteira mista, onde deseja-se escolher $p$ estilos de refer\^encia (medianas) de forma a minimizar a dist\^ancia total entre todos os estilos e seu centro de refer\^encia mais pr\'oximo.

\subsubsection{Formula\c{c}\~ao Matem\'atica}

\begin{equation}
\text{Minimize } \sum_{i=1}^{n} \sum_{j=1}^{n} d_{ij} x_{ij}
\end{equation}

Sujeito a:

\begin{equation}
\sum_{j=1}^{n} y_j = p
\end{equation}

\begin{equation}
\sum_{j=1}^{n} x_{ij} = 1 \quad \forall i
\end{equation}

\begin{equation}
x_{ij} \leq y_j \quad \forall i,j
\end{equation}

\begin{equation}
x_{ij}, y_j \in \{0,1\}
\end{equation}

onde $d_{ij}$ representa a dist\^ancia entre os estilos $i$ e $j$, $y_j = 1$ indica que o estilo $j$ \'{e} uma mediana, e $x_{ij} = 1$ indica que o estilo $i$ est\'a associado \`a mediana $j$.

\subsubsection{Etapas da Implementa\c{c}\~ao}

\begin{enumerate}
    \item Constru\c{c}\~ao da matriz de dist\^ancia sensorial entre estilos;
    \item Defini\c{c}\~ao do n\'umero de medianas $p$;
    \item Aplica\c{c}\~ao do modelo p-mediana via pacote \texttt{JuMP.jl} com \texttt{GLPK} como solver;
    \item Identifica\c{c}\~ao dos estilos medianas e agrupamento de estilos por proximidade.
\end{enumerate}

\subsection{Implementa\c{c}\~ao em Julia}

Foram utilizados os seguintes pacotes Julia:

\begin{itemize}
\item \texttt{DataFrames.jl} -- manipula\c{c}\~ao de dados tabulares
\item \texttt{LinearAlgebra.jl} -- opera\c{c}\~oes vetoriais e matriciais
\item \texttt{Statistics.jl} -- normaliza\c{c}\~ao e estat\'isticas descritivas
\item \texttt{CSV.jl} -- leitura de arquivos CSV
\item \texttt{JuMP.jl} -- modelagem de otimiza\c{c}\~ao
\item \texttt{GLPK.jl} -- resolvedor de programa\c{c}\~ao inteira
\end{itemize}

O projeto foi estruturado em m\'odulos conforme apresentado a seguir:

\begin{lstlisting}[float,language=Python, caption=Estrutura modular do c\'odigo Julia]
module BeerSimilarity
    using DataFrames, LinearAlgebra, Statistics, CSV, JuMP, GLPK

    include("data_loader.jl")
    include("normalization.jl")
    include("distance_metrics.jl")
    include("p_median_solver.jl")

    export compute_distance_matrix, solve_p_median
end
\end{lstlisting}

\subsection{Valida\c{c}\~ao e Visualiza\c{c}\~ao}

A valida\c{c}\~ao dos agrupamentos gerados baseou-se em crit\'erios de coer\^encia sensorial:

\begin{itemize}
\item Coes\~ao intra-cluster (cervejas pr\'oximas entre si);
\item Separabilidade entre clusters distintos;
\item Conformidade com classifica\c{c}\~oes da Brewers Association (2024).
\end{itemize}

Para visualiza\c{c}\~ao dos resultados, foram utilizadas:

\begin{itemize}
\item Mapas de calor da matriz de dist\^ancia (Figura~\ref{fig:heatmap});
\item Gr\'aficos de agrupamento com base nas medianas (Figura~\ref{fig:clusters});
\item Tabela com estilos medianas identificados e seus respectivos grupos (Tabela~\ref{tab:medianas}).
\end{itemize}

\begin{figure}[H]
\centering
\includegraphics[width=0.75\textwidth]{heatmap.png}
\caption{Mapa de calor da matriz de dist\^ancia sensorial entre estilos}
\label{fig:heatmap}
\end{figure}

\begin{figure}[H]
\centering
\includegraphics[width=0.75\textwidth]{clusters.png}
\caption{Gr\'afico de agrupamento de estilos por p-mediana ($p=5$)}
\label{fig:clusters}
\end{figure}

\begin{table}[H]
\centering
\caption{Estilos medianas e agrupamentos resultantes}
\label{tab:medianas}
\begin{tabular}{ll}
\toprule
\textbf{Estilo Mediana} & \textbf{Estilos Associados} \\
\midrule
IPA Americana & Session IPA, Double IPA, West Coast IPA \\
Weissbier & Dunkelweizen, Kristallweizen, Hefeweizen \\
Stout Irlandesa & Dry Stout, Foreign Extra, Oatmeal Stout \\
Pilsner Alem\~a & Helles, Kellerbier, Zwickelbier \\
Saison & Bière de Garde, Farmhouse Ale, Grisette \\
\bottomrule
\end{tabular}
\end{table}

\section{RESULTADOS E DISCUSSÕES}

Os testes realizados com perfis simulados demonstraram a eficácia do sistema de recomendação desenvolvido. 

\subsection{Análise dos Perfis de Usuários}

Os resultados dos testes com diferentes perfis de usuários estão sumarizados na Tabela.

\subsection{Evolução da Precisão}

\subsection{Análise dos Resultados}

Os resultados obtidos confirmam a eficácia do sistema de recomendação desenvolvido. Observou-se que:

\subsection{Desempenho Computacional}

\subsection{Limitações Identificadas}

Durante os testes, foram identificadas algumas limitações do sistema atual:

\subsection{Comparação com Métodos Tradicionais}

% ========== CONCLUSÃO ==========
\newpage
\section{CONCLUSÃO}

\subsection{Objetivos Alcançados}

Todos os objetivos propostos foram satisfatoriamente atingidos:

\subsection{Contribuições do Trabalho}

O trabalho oferece contribuições significativas em múltiplas dimensões:

\subsubsection{Contribuições Técnicas}

\subsubsection{Contribuições Metodológicas}

\subsubsection{Contribuições Práticas}


\subsection{Limitações e Trabalhos Futuros}

\subsubsection{Limitações Atuais}


\subsubsection{Propostas de Trabalhos Futuros}

\subsection{Considerações Finais}

% ========== REFERÊNCIAS ==========
\newpage
\section{REFERÊNCIAS}

\begin{thebibliography}{20}

\bibitem{aggarwal2016}
AGGARWAL, Charu C. \textbf{Recommender Systems: The Textbook}. 1. ed. Cham: Springer International Publishing, 2016. 498 p.

\bibitem{bezanson2017}
BEZANSON, Jeff; EDELMAN, Alan; KARPINSKI, Stefan; SHAH, Viral B. Julia: a fresh approach to numerical computing. \textbf{SIAM Review}, Philadelphia, v. 59, n. 1, p. 65-98, 2017.

\bibitem{brewers2024}
BREWERS ASSOCIATION. \textbf{Beer Style Guidelines}. Boulder: Brewers Association, 2024.


\bibitem{bonatto2024}
BONATTO, Diego. A new classification system of beer categories and styles based on large-scale data mining and self-organizing maps of beer recipes. \textbf{Journal of Food Science}, Hoboken, v. 89, n. 3, p. 1245-1260, 2024.

\bibitem{carreira2018}
CARREIRA, João Pedro Sousa. \textbf{Algoritmos certificadores: fundamentos e aplicações}. 2018. 156 f. Dissertação (Mestrado em Ciência da Computação) -- Faculdade de Ciências, Universidade de Lisboa, Lisboa, 2018.

\bibitem{mapa2020}
MAPA -- MINISTÉRIO DA AGRICULTURA, PECUÁRIA E ABASTECIMENTO. \textbf{Anuário da Cerveja 2019}. Brasília: MAPA, 2020. Disponível em: https://www.gov.br/agricultura/pt-br/assuntos/inspecao/produtos-vegetal/publicacoes/anuario-da-cerveja-2019. Acesso em: 4 jun. 2025.

\bibitem{mcauley2012}
MCAULEY, Julian; LESKOVEC, Jure; JURAFSKY, Dan. Learning attitudes and attributes from multi-aspect reviews. In: \textbf{IEEE 12th International Conference on Data Mining (ICDM)}. Brussels: IEEE, 2012. p. 1020-1025.

\bibitem{mcauley2013}
MCAULEY, Julian; LESKOVEC, Jure. From amateurs to connoisseurs: modeling the evolution of user expertise through online reviews. In: \textbf{Proceedings of the 22nd International Conference on World Wide Web (WWW)}. Rio de Janeiro: ACM, 2013. p. 897-908.

\bibitem{pereira2021}
PEREIRA, Diogo Costa. Reconhecendo estilos de cerveja com uma rede neural artificial. \textbf{RECIMA21 -- Revista Científica Multidisciplinar}, São Paulo, v. 2, n. 4, p. e24200, 2021. Disponível em: https://www.recima21.com.br/index.php/recima21/article/view/24200. Acesso em: 4 jun. 2025.

\bibitem{pereira2024}
PEREIRA, Diogo Costa. \textbf{Cerveja e Machine Learning: recomendando estilos de cerveja}. 2024. 89 f. Trabalho de Conclusão de Curso (Bacharelado em Ciência da Computação) -- Instituto de Computação, Universidade Federal Fluminense, Niterói, 2024.

\bibitem{pirkola1998}
PIRKOLA, Ari. The effects of query structure and dictionary setups in dictionary-based cross-language information retrieval. In: \textbf{SIGIR '98: Proceedings of the 21st Annual International ACM SIGIR Conference on Research and Development in Information Retrieval}. Melbourne: ACM, 1998. p. 55-63.

\bibitem{ricci2015}
RICCI, Francesco; ROKACH, Lior; SHAPIRA, Bracha. \textbf{Recommender Systems Handbook}. 2. ed. New York: Springer Science+Business Media, 2015. 1006 p.

\end{thebibliography}

% ========== ANEXOS ==========
\newpage
\section{ANEXOS}

\subsection{Anexo A -- Questionário de Preferências}

\subsubsection{Questionário Estruturado para Coleta de Preferências Sensoriais}

\textbf{Instruções:} Avalie cada atributo sensorial de acordo com sua preferência em cervejas, utilizando a escala de 1 a 10, onde:
\begin{itemize}
\item 1 = Não gosto nada / Evito completamente
\item 5 = Indiferente / Neutro
\item 10 = Gosto muito / Procuro especificamente
\end{itemize}

\begin{table}[H]
\centering
\begin{tabular}{|l|p{8cm}|c|}
\hline
\textbf{Atributo} & \textbf{Descrição} & \textbf{Nota (1-10)} \\
\hline
Doçura & Presença de açúcares residuais, sabor adocicado & \underline{\hspace{1cm}} \\
\hline
Amargor & Intensidade do amargor proveniente do lúpulo & \underline{\hspace{1cm}} \\
\hline
Corpo & Sensação de "peso" e textura na boca & \underline{\hspace{1cm}} \\
\hline
Frutado & Aromas e sabores que lembram frutas & \underline{\hspace{1cm}} \\
\hline
Maltado & Sabores provenientes do malte (cereal, pão) & \underline{\hspace{1cm}} \\
\hline
Tostado & Sabores de torrefação, café, chocolate & \underline{\hspace{1cm}} \\
\hline
Cítrico & Aromas cítricos (limão, laranja, grapefruit) & \underline{\hspace{1cm}} \\
\hline
Floral & Aromas florais delicados & \underline{\hspace{1cm}} \\
\hline
\end{tabular}
\end{table}

\textbf{Informações Complementares:}

\begin{itemize}
\item Experiência com cervejas artesanais: \underline{\hspace{4cm}}
\item Frequência de consumo: \underline{\hspace{4cm}}
\item Estilos conhecidos e apreciados: \underline{\hspace{6cm}}
\item Restrições alimentares: \underline{\hspace{4cm}}
\end{itemize}

\subsection{Anexo B -- Código Fonte Julia}

\subsubsection{Módulo Principal (main.jl)}

\begin{lstlisting}[language=Python, caption=Código principal do sistema de recomendação]
module BeerRecommendation

using DataFrames, LinearAlgebra, Statistics, CSV

# Estrutura para representar uma cerveja
struct Beer
    name::String
    style::String
    abv::Float64
    ibu::Float64
    srm::Float64
    sweetness::Float64
    bitterness::Float64
    body::Float64
    fruity::Float64
    malty::Float64
    roasted::Float64
    citrus::Float64
    floral::Float64
end

# Estrutura para representar preferências do usuário
struct UserPreferences
    sweetness::Float64
    bitterness::Float64
    body::Float64
    fruity::Float64
    malty::Float64
    roasted::Float64
    citrus::Float64
    floral::Float64
end

# Função para normalizar os dados
function normalize_data(data::Matrix{Float64})
    normalized = copy(data)
    for i in 1:size(data, 2)
        col = data[:, i]
        μ = mean(col)
        σ = std(col)
        normalized[:, i] = (col .- μ) ./ σ
    end
    return normalized
end

# Função para calcular similaridade por produto escalar
function cosine_similarity(user_prefs::Vector{Float64}, 
                          beer_features::Vector{Float64})
    return dot(user_prefs, beer_features) / 
           (norm(user_prefs) * norm(beer_features))
end

# Função para calcular distância euclidiana
function euclidean_distance(user_prefs::Vector{Float64}, 
                           beer_features::Vector{Float64})
    return norm(user_prefs - beer_features)
end

# Função principal de recomendação
function recommend_beers(beers::Vector{Beer}, 
                        user_prefs::UserPreferences, 
                        n_recommendations::Int=5)
    
    # Extrair características das cervejas
    beer_features = Matrix{Float64}(undef, length(beers), 8)
    for (i, beer) in enumerate(beers)
        beer_features[i, :] = [beer.sweetness, beer.bitterness, 
                              beer.body, beer.fruity, beer.malty, 
                              beer.roasted, beer.citrus, beer.floral]
    end
    
    # Normalizar dados
    normalized_features = normalize_data(beer_features)
    
    # Converter preferências do usuário para vetor
    user_vector = [user_prefs.sweetness, user_prefs.bitterness,
                   user_prefs.body, user_prefs.fruity, user_prefs.malty,
                   user_prefs.roasted, user_prefs.citrus, user_prefs.floral]
    
    # Calcular similaridades
    similarities = Vector{Float64}(undef, length(beers))
    for i in 1:length(beers)
        similarities[i] = cosine_similarity(user_vector, 
                                          normalized_features[i, :])
    end
    
    # Ordenar por similaridade
    sorted_indices = sortperm(similarities, rev=true)
    
    # Retornar top N recomendações
    recommendations = beers[sorted_indices[1:n_recommendations]]
    scores = similarities[sorted_indices[1:n_recommendations]]
    
    return recommendations, scores
end

# Função para atualizar preferências com feedback
function update_preferences(current_prefs::UserPreferences,
                           feedback_beer::Beer,
                           rating::Float64,
                           learning_rate::Float64=0.1)
    
    # Ajustar preferências baseado no feedback
    adjustment = (rating - 5.0) * learning_rate  # 5.0 é neutro
    
    new_sweetness = current_prefs.sweetness + 
                   adjustment * (feedback_beer.sweetness - current_prefs.sweetness)
    new_bitterness = current_prefs.bitterness + 
                    adjustment * (feedback_beer.bitterness - current_prefs.bitterness)
    new_body = current_prefs.body + 
              adjustment * (feedback_beer.body - current_prefs.body)
    new_fruity = current_prefs.fruity + 
                adjustment * (feedback_beer.fruity - current_prefs.fruity)
    new_malty = current_prefs.malty + 
               adjustment * (feedback_beer.malty - current_prefs.malty)
    new_roasted = current_prefs.roasted + 
                 adjustment * (feedback_beer.roasted - current_prefs.roasted)
    new_citrus = current_prefs.citrus + 
                adjustment * (feedback_beer.citrus - current_prefs.citrus)
    new_floral = current_prefs.floral + 
                adjustment * (feedback_beer.floral - current_prefs.floral)
    
    return UserPreferences(new_sweetness, new_bitterness, new_body,
                          new_fruity, new_malty, new_roasted,
                          new_citrus, new_floral)
end

export Beer, UserPreferences, recommend_beers, update_preferences

end # module
\end{lstlisting}

\subsubsection{Exemplo de Uso do Sistema}

\begin{lstlisting}[language=Python, caption=Exemplo de uso do sistema de recomendação]
# Carregar o módulo
using BeerRecommendation

# Criar base de dados de exemplo
beers = [
    Beer("American IPA", "IPA", 6.5, 65, 8, 3, 9, 6, 4, 5, 2, 8, 3),
    Beer("Belgian Dubbel", "Belgian", 7.2, 20, 15, 8, 4, 7, 6, 8, 3, 2, 4),
    Beer("Dry Stout", "Stout", 4.8, 35, 35, 2, 6, 8, 3, 6, 9, 1, 1),
    Beer("German Pilsner", "Pilsner", 5.0, 30, 3, 4, 5, 4, 2, 6, 1, 3, 5),
    Beer("Witbier", "Wheat", 4.5, 15, 3, 6, 2, 3, 7, 4, 1, 4, 8)
]

# Criar perfil de usuário que gosta de cervejas doces e frutadas
user_prefs = UserPreferences(8, 3, 5, 8, 6, 2, 4, 7)

# Obter recomendações
recommendations, scores = recommend_beers(beers, user_prefs, 3)

# Exibir resultados
println("Recomendações para o usuário:")
for (i, (beer, score)) in enumerate(zip(recommendations, scores))
    println("$i. $(beer.name) - Similaridade: $(round(score, digits=3))")
end

# Simular feedback e atualizar preferências
feedback_beer = recommendations[1]  # Primeira recomendação
rating = 8.0  # Usuário gostou muito

# Atualizar preferências
updated_prefs = update_preferences(user_prefs, feedback_beer, rating)

# Obter novas recomendações
new_recommendations, new_scores = recommend_beers(beers, updated_prefs, 3)

println("\nNovas recomendações após feedback:")
for (i, (beer, score)) in enumerate(zip(new_recommendations, new_scores))
    println("$i. $(beer.name) - Similaridade: $(round(score, digits=3))")
end
\end{lstlisting}

\subsection{Anexo C -- Dados da Base de Cervejas}

\subsubsection{Amostra da Base de Dados Completa}

\begin{table}[H]
\centering
\scriptsize
\begin{tabular}{|l|c|c|c|c|c|c|c|c|c|c|c|}
\hline
\textbf{Estilo} & \textbf{ABV} & \textbf{IBU} & \textbf{SRM} & \textbf{Doç} & \textbf{Amg} & \textbf{Corp} & \textbf{Frut} & \textbf{Malt} & \textbf{Tost} & \textbf{Cít} & \textbf{Flor} \\
\hline
American IPA & 6.5 & 65 & 8 & 3 & 9 & 6 & 4 & 5 & 2 & 8 & 3 \\
Belgian Dubbel & 7.2 & 20 & 15 & 8 & 4 & 7 & 6 & 8 & 3 & 2 & 4 \\
Dry Stout & 4.8 & 35 & 35 & 2 & 6 & 8 & 3 & 6 & 9 & 1 & 1 \\
German Pilsner & 5.0 & 30 & 3 & 4 & 5 & 4 & 2 & 6 & 1 & 3 & 5 \\
Witbier & 4.5 & 15 & 3 & 6 & 2 & 3 & 7 & 4 & 1 & 4 & 8 \\
Porter & 5.8 & 28 & 25 & 4 & 5 & 7 & 2 & 7 & 7 & 1 & 2 \\
Saison & 6.0 & 22 & 5 & 3 & 4 & 4 & 6 & 5 & 1 & 3 & 6 \\
Hefeweizen & 5.2 & 12 & 4 & 5 & 2 & 5 & 5 & 6 & 1 & 2 & 4 \\
Imperial Stout & 9.5 & 45 & 40 & 3 & 7 & 9 & 3 & 7 & 9 & 1 & 1 \\
Pale Ale & 5.5 & 35 & 6 & 4 & 6 & 5 & 3 & 6 & 2 & 5 & 3 \\
\hline
\end{tabular}
\end{table}

\textbf{Legenda:} Doç = Doçura, Amg = Amargor, Corp = Corpo, Frut = Frutado, Malt = Maltado, Tost = Tostado, Cít = Cítrico, Flor = Floral

\end{document}
